\section{Algorithm}
\textbf{Skeleton Partition.} Let $M$ denote the mesh model, and let S denote the Laplacian skeleton obtained via \cite{AuTCCL08}. We propose an algorithm for partitioning S into a minimum set of \emph{disjoint} subgraphs, each of which can be fabricated in a 3D printer without using support structures. Decomposing S into two pieces can be done by duplicating a node $v$ multiple times in accordance to the number of the graph edges (arcs) incident to $v$; and, partitioning $M$ at node $v$ requires the determination of the position and normal of a cutting plane. To guarantee an aesthetical look on the resulting surface with shortest seams, we need a constraint to minimize the peripheral length of the cut in terms of the position and normal of the cutting plane. Once the cutting plane is determined, it affects the printing direction and thus the shape of the subgraph. Hence, this is an essential \emph{chicken-and-egg} problem. In addition, we also need to consider the printing volume constraint, i.e., the volume of the printing model should be within the working volume of a given 3{D} printer.

To summarize, our objectives are the minimization of (1) the number of partitioned components $N$; (2) the total peripheral length of each cut $L_i$, i.e., $\Sigma L_i$. The constraints of the problem are as follows:
(i) Each arc of the partitioned subgraph $H_i$ subtends to an axis by an angle of no larger than $\theta$, where $\theta$ is defined based on printing experiments without using any support structure. This guarantees that the corresponding mesh component is support-free during the printing process;
(iii) Let the normal direction of the cut $c$ (pointing to the exterior of the partitioned part) be denoted as $n(c)$; as the directed arcs of $H_i$ are translated to a common origin, they form a fork, the fork has a central ray, which is the middle axis of the minimum cone that encloses the fork, let $cone(H_i)$ be the cone, let the central ray of $cone(H_i)$ be denoted as $r(H_i)$; in order to guarantee support-free fabrication for the boundary of the cut, the angle between $n(c)$ and $r(H_i)$ should satisfy $angle(n(c), r(Hi)) \leq \pi/2 + \theta$;
(iv) The base of a printing model should be large enough to gather sticky force from the building platform, such that the model is not deformed during the building process. Formally, let $b(H_i)$ denote the area of the base of the mesh component corresponding to $H_i$, let $\tau$ be a user-defined threshold value., then we have $b(H_i) \geq \tau$. Here $\tau$ can be determined empirically;
(v) Each cut partitions a single subgraph.

We have the following optimization system:
\begin{equation*}
\begin{aligned}
& \underset{x}{\text{minimize}} \quad N \quad \text{and} \quad \sum_{i=1}^N{L_i},
\quad \text{subject to:} \\
& A(e, r(H_i)) \leq \theta, \; i = 1, \ldots, m, \forall e \in H_i, & (1)\\
& A(n(c), r(H_i)) \leq \pi/2+ \theta, & (2)\\
& b(H_i) \geq \tau, & (3)\\
& c \cap S = c \cap H_i, & (4)
\end{aligned}
\end{equation*}
where all $H_i$-s constitute a partition of the original graph to be cut off. A direct exploration of all possible partitions over the graph $G$ could quickly leads to exponential complexity. The key here is to quickly explore potential good partitions in a way that subsequent exploration of the graph is limited to those which leads to a less value of the target optimization. Hence, we employ a randomized exploration algorithm based on Monte Carlo.

The idea is to minimize the two target terms (number of subgraphs and the total cutting length) sequentially rather than simultaneously. To minimize the number of subgraphs, we use a greedy strategy. In particular, assume that we are given a function $Tirm\_BFS(v, G, \theta)$ which traverses $G$ from $v$ in a breath first search manner until all arcs satisfying constraints (2-3) are determined. The following algorithm sketches the idea of the skeleton decomposition. The main idea is to randomly search for maximal candidate subgraphs using Monte Carlo Method, which randomly chooses a node of $G$ to start traversing and randomly grows the subgraph w.r.t. the constraints (1-3).

\begin{algorithm}
\caption{$SkeletonMeshDecomposition(S, M)$}
\label{alg:Framwork}
\begin{algorithmic}[1]
\REQUIRE~
The Laplacian skeleton $S$ of a mesh model $M$;
\ENSURE~
The decomposition of $S$ into a set of the least pieces of subgraphs $T$, each arc of which subtends to an axis by an angle of no larger than $\theta$;
\STATE $T = \emptyset$; $min = \inf$; $count$ = 0; $max_iter$ = a user defined large constant;
\WHILE {$count < max_iter$}
\STATE  $G=S$; $U= \emptyset$;
\WHILE {$G\neq \emptyset$}
\FOR{$i=1$; $i<| S |$; $i++$ }
\STATE $H = Trim\_BFS(v_i, G, \theta)$;
\STATE $H = S / H$;
\STATE $U = U \cup H$;
\IF {$| U | < min$}
\STATE $T = U$;
\STATE  $min = | U | $;
\ENDIF
\ENDFOR
\ENDWHILE
\STATE $count =count + 1$;
\ENDWHILE
\label{code:fram:select} \\
\RETURN $T$;
\end{algorithmic}
\end{algorithm}


Next we shall show how $Tirm\_BFS(v, G, \theta)$ works to find a maximal subgraph starting at $v$ that satisfies the angle constraint. Let $H$ be the current subgraph obtained so far. When an arc $e$ of $G$ is visited, we need to determine whether it should be included into $H$. If the start of each outgoing arc of $H$ is moved to a common origin, then the arcs form a fork of rays (Figure QAZ). A naive method to judge whether $e$ should be included is to move the start of $e$ to the origin of the fork, and compute the angle between $e$ and each arc of the fork, $e$ is included if the maximum angle between $e$ and each arc of the fork does not exceed $\pi/2$. However, this method would require $O(K^{2})$ time, where $K$ is the number of the nodes of $S$. To speed up this process, we keep the pair of vectors hat form the largest angle and judge whether a new vector expands the angle of the fork; if so, determine the other vector (may not be an arc of $H$). See in Figure QAZ, let $\hat{e_i}$  and $\hat{e_j}$ be the units of these two vectors obtained so far. For simplicity, we denoted by $F( \hat{e_i}, \hat{e_j} )$ the fork with the starts of all unit vectors converging at the origin of the coordinate frame, where $\hat{e_i}$  and  $\hat{e_j}$  are the pair of unit vectors that form the largest angle in the fork. Let $\hat{e_k}$  be the unit of a new vector to be processed next, if $\hat{e_k}$ penetrates through the blue circle, then no change need to be made to the fork; otherwise, let $D_{i,j}$ denote the spherical disk that passes through the endpoints of $\hat{e_i}$  and $\hat{e_j}$ whose central axis is collinear with $\hat{e_i}$  + $\hat{e_j}$, let $c_{i,j}$ be the center of $D_{i,j}$, let $B_(i,j)$ be the boundary circle of $D_{i,j}$. The circle passing through $\hat{e_k}$  and $c_{i,j}$, denoted as $O( \hat{e_k}, c_{i,j})$, intersects  $B_{i,j}$ at two points, let $q$ the point further away from the endpoint of $\hat{e_k}$, then $\rightharpoonup{oq}$ and $\hat{e_k}$ are the two extreme vectors that to be used in the next iteration. To summarize,
A new arc $e_k$ is taken by Function $Tirm\_BFS$ if and only if one of the following two conditions is met:
(1) the angle between $\rightharpoonup{oc_{i,j}}$ and $\hat{e_k}$, denoted as $A(\rightharpoonup{oc_{i,j}}, \hat{e_k})$, satisfies  $A(\rightharpoonup{oc_{i,j}}, \hat{e_k}) \leq A(\rightharpoonup{oc_{i,j}}, \hat{e_i})$;
(2) $A(\rightharpoonup{oc_{i,j}}, \hat{e_k})  \leq \pi - 2\theta$, where $q = O( \hat{e_k}, c_{i,j}) \cap B_{i,j}$.


\textcolor{red}{Figure QAZ: Illustration of unit vectors, unit sphere, spherical disks, and the determination of taking a new edge in $Trim\_BFS$.}

\textbf{Mesh Partition.} The skeleton partition tells us a rough sketch of the mesh partition, i.e., the cutting plane should be in the vicinity of each node $v$ incident to two distinct subgraphs. Yet we need to determine the exact positions and orientations of the cutting planes. For each node $v$ that is incident to at least two distinct subgraphs $H_i$ and $H_j$, we process it using the following cutting principle:

In Figure TJK, at the position of node $v$, we define two fillets, each of which contains all planes that orthogonal to the vectors in $cone(H_i)$ or $cone(H_j)$ and passing through $v$, denoted as $fillet(H_i)$ and $fillet(H_j)$ respectively. Depending on the relative position of $fillet(H_i)$ and $fillet(H_j)$, we have the following cases: $(fillet(Hi) \cap fillet(Hj)) - \{v\} \neq \emptyset$;
Then randomly sample a set of cutting planes from $(fillet(Hi) \cap Fillet(Hj)) - \{v\}$, and determine the one achieving the minimum cutting length.
In this case, two cuts $c_1$ and $c_2$ are inevitable in order to separate the mesh, however, care must be taken as the angle between $c_1$ and $c_2$ should be constrained by $A(n(c_1), n(c_2)) \leq \pi/2+ \theta$. If this constraint is violated, one more cut in between $c_1$ and $c_2$ is required; while the base of each partitioned component should be constrained by $b(H_i) \geq \tau$ and $b(H_j) \geq \tau$. If any of the constraints is not satisfied, we shall translate the fillets along the build direction in an opposite sense until the constraints are satisfied (See Figure TJK). In this process, if the translation hits a induce


\textcolor{red}{Figure TJK: Illustration of two fillets incident at a common node.}

Let $R(v)$ be the set of reflex vertices on M that are incident to v during the Laplacian shrinking process \cite{AuTCCL08}, we shall truncate $R(v)$ such that the non-significant reflex vertices are removed away. Here, given a vertex vi and any of its neighbor $v_j$, $v_i$ is reflex if $(v_i - v_j)(n_j - n_i)$ is nonnegative for some $j$, we quantify the significance of a reflex vertex as the magnitude of $(v_i - v_j)(n_j - n_i)$ \cite{au2012mesh}.





For each vertex in $R(v)$, we shall process a pair of fillets as done for node $v$ above. Particularly, if $v$ is neighbor to at least three nodes in $S$ while is only neighbored to a single node $w$ in subgraph $H_i$, then the separation of the mesh component corresponding to $H_i$ may not be feasible if only $R(v)$ is used, this is because the position of $v$ is designed for the purpose of connecting the different branches. To separate the mesh component of $H_i$, we process arc $vw$ as follows:
Given a user-defined threshold value $\delta$, if $| vw | \leq \delta$, then $R(v)$ is extended to contain the reflex vertices of $M$ that are incident to $w$. On the other hand, if $| vw | > \delta$, then a partition of the arc $vw$ by an interval of $\delta$ is applied. Subsequently, for each node along the arc $vw$, the above scheme for processing a pair of fillets for node $v$ is exploited, see figure GHJ.
Finally, if the cut-subgraph constraint (Eq. (5)) is violated, a sequence of iterative cutting is required.


\textcolor{red}{Figure GHJ: 2D illustration editing a long arc for cutting.}

Cut-subgraph Intersection Detection: In order to take care of Eq. (7), the intersections of a cut and the subgraphs can be efficiently determined by taking advantage of the correspondence between the mesh vertices and the skeleton nodes: each vertex of $M$ is mapped to a single node of $S$. Therefore, as a cut goes through the mesh surface, the endpoints of the cut edges of $M$ lend us the information of the cut subgraphs. Approximately, if a cut $c$ goes into an edge whose endpoints are incident to a subgraph $H$, then $c$ cuts $H$.



\begin{algorithm}
\caption{Algorithm: $Tirm_BFS(v, S,\theta)$}
\label{alg:Framwork}
\begin{algorithmic}[1]
\REQUIRE~
A node $v$ of Laplacian skeleton $S$, an angular value $\theta$;
\ENSURE~
 A maximal subgraph $H$ rooted at $v$ and its corresponding mesh component that meet the constraints (Eq.2-5);
\STATE starting from $v$, initialize F($\hat{e_i}$, $\hat{e_j}$), $H = /emptyset$;
\WHILE  {the current arc $e_k$ of $S$ picked by the BFS process is nonempty}
  \IF  {$A(\vec a\quad\overrightarrow(o, c_(i,j)),  \hat{e_k}) \leq A(\vec a\quad\overrightarrow(o, c_(i,j)),  \hat{e_i})$}
  \STATE  $H = H \cup {e_k}$;
  \ENDIF
  \STATE $q = O( \hat{e_k}, c_(i,j)) \cap B_(i,j)$;
  \IF  {$A(\vec a\quad\overrightarrow(o, q),  \hat{e_k}) �� \pi- 2\theta$}
  \STATE  $\hat{e_i}=  \hat{e_k}$;
  \STATE $\hat{e_j} =  \vec a\quad\overrightarrow(o, q)$;
  \STATE  update $B_(i,j)$ and $c_(i,j)$;
  \STATE  $H = H \cup {e_k}$;
  \ENDIF
\ENDWHILE
\STATE  call the cutting scheme for $M$;
\STATE  $M= M/M_H$;
\label{code:fram:select} \\
\RETURN  $H$ and $M_H$;
\end{algorithmic}
\end{algorithm}


In line 2 of $Tirm\_BFS$, the BFS process randomly chooses an arc incident to $v$ to proceed on. In order to guarantee a greater chance of converging to the optimal result in a short time, we apply a training-and-learning procedure for the first 1000 runs. Formally, let $N_v$ be the number of times an arc is chosen as the exit arc when node $v$ is visited. Given the data of the first 1000 runs, as a node $v$ is visited, the probability of choosing an arc $e$ as an outgoing arc in the subsequent runs is,

	$P(v, e) = N_v/1000$	(7)

To further speed up the process of $Trim_BFS$, we assign a mark that stores the minimum number of subgraphs obtained so far, such that the current branching can be terminated if its output number of subgraphs is larger than the mark.

Since the traversing process assigns a specific direction to each arc that was originally undirected in $S$, it is not obvious whether the angle constraint is satisfied, To clarify this, we provide the following lemma.

Lemma 1: $H = Tirm_BFS(v, G, \theta)$ is a maximal subgraph of $G$ that satisfies the angle constraint, i.e., each arc of $H$ subtends to an axis by an angle of no larger than $\theta$.

Proof: Suppose to the contrary that $H$ violates the angle constraint, there exists a directed arc that does not satisfy the angle constraint. For example, arc $\vec a\quad\overrightarrow(c, a)$ or arc $\vec a\quad\overrightarrow(b, c)$ in Figure WSX. Such case is impossible as Line 3 and Line 6 of function $Trim_BFS$ excludes any directed arc that violates the angle constraint of no larger than $\theta$ with respect to the (virtual) central axis.
It remains to prove that $H$ is maximal, i.e., the largest graph rooted at $v$ that covers all the arcs that satisfies the angle constraint. Suppose that this is not true, there must exist an arc that was mistakenly discarded due to the direction in which the arc is traversed. Let $(b, c)$ be one of such arcs, as illustrated in Figure WSX (a). When the arc is directed from $b$ to $c$, it is not included as it violates the angle constraint, but can be included if the arc directed from $c$ to $b$. We shall prove in a case-by-case basis.
If $c$ is not reachable from $v$ via a directed path passing through $b$ (Figure WSX (c)), then $c$ is only reachable from $b$, arc $bc$ should not be included and line 6 of Function $Trim_BFS$ correctly handle this case.
Otherwise, $c$ is reachable from $v$ via a directed path without passing through $b$ (Figure WSX (d)). As $c$ is visited, by Line 2 of Function $Trim_BFS$, each arc leaving $c$ is considered, and $\vec a\quad\overrightarrow(c, b)$ is correctly included into $H$. This completes the proof. ��



                   (a)                      (b)

Figure WSX: Illustration of the inclusion of a directed arc into a maximal subgraph by $Trim\_BFS$.




