\section{Introduction}

3D printing, or additive manufacturing, has drawn growing interests from researchers in computer graphics. Fused deposition modeling (FDM), stereolithographic (SLA), Selective Laser Melting (SLM) and Selective Laser Sintering (SLS) and the four most popular means of 3D printing techniques. Although 3D printing has seen its applications in producing arbitrarily intricate 3D models, the price of the printing materials, especially for those with high quality, are still outrageously high. Therefore, it is desirable to reduce materials used in the fabrication process. Note that this is also a critical operation for reducing production time and therefore the total production cost. For this purpose, an efficient method is the minimizing of support structures, which are removed in the post-processing phase of the fabrication task.

As for minimizing support structures, Autodesk R MeshMixerTM provides a semiautomatic orientation optimization tool to minimize support volume, support area, structural strength, or a combination of these three attributes. However, it requires the operators�� experience in setting the geometric parameters manually. A number of literatures have studied various factors that influence the volume of supports, e.g., optimizing the topology of the support structure \cite{DumasHL14,VanekGB14}, determining an optimal fabrication direction \cite{Zhang:2015,HildebrandBA13,padhye2011multi}; partitioning any given model into a set of separate parts that preserve nice geometric features \cite{VanekGBMCSM14,SongFLF15}.

However, the partition results of these methods do not simultaneously respect the geometric features and support-free printability of the portioned parts. Further, no existing algorithms ever consider the problem of partitioning a model into the least number of pieces whose fabrication is free of support structures. The least number of portioned parts corresponds to the least number of seams on the final assembled model, which means a nice aesthetics preservation of the model surface; and the support-free fabrication means saving material to the most extent, which is very meaningful in reducing fabrication cost since the prices of printing materials (metal powder, resin, nice plastics, etc.) are still outrageously high. This motivates us to explore an efficient algorithm for solving the issue.
Our solution of the problem draws inspiration from the skeleton of organic models: the topology changes of a natural model can be determined by its skeleton, and each chunk of a mesh corresponding to a segment of the skeleton; further, the mesh chunk is a cylinder-like shape which can be printed free of support structures if the printing direction is parallel to the skeleton.

Automatically partitioning 3D models into components that are consistent with human perception is an extremely intricate due to the lack of semantic information. Therefore, we restrict our focus on organic or man-made models whose shapes are well-defined by their skeletons. Further, support structures are required for the interior and exterior surface of a mesh model during the 3D printing process. For simplicity, our approach assumes that the interior of a mesh model is fully filled, usually the infill can be set as a grid whose porosity can be varied by users, as allowed in almost all existing commercial 3D printing software. This is very useful in preserving the surface quality of any printed object. Therefore, our objective is to partition a model according to the growth of its skeleton. Formally, given a 3D printer, if a facet subtends to a common axis by an angle of less than or equal to, then the facet can be printed without using any support structure. This inspires us to compute a minimum set of subgraph of the skeleton, such that each arc in any subgraph subtends to a common axis by an angle of less than or equal to ?, the corresponding chunk of the mesh is therefore support-free as printed along this axis. In general, a cone of axes satisfies the angle constraint. However, the volume of the chunk should be within the working space of the printer. Further, if the center of mass of the chunk diverts from the support center too much, e.g., the gravitational torque applied at the mass center is too far away from the center of support that it bends the printing model by a layer of thickness, then the surface quality of the model is poor or the printing task fails since the next printing layer cannot be firmly attached to the previous layer.

Our method makes the following contributions:

\begin{itemize}
\item {We partition any given mesh model into the least number of parts that are printable free of support structures; meanwhile, we preserve the aesthetics of the surface finish with the least number of seams whose length is minimized.}
\item {We propose a partition method based on the guide of 1D Laplacian skeleton of any given mesh model.}
\end{itemize}