\section{Introduction}

3D printing, or additive manufacturing, has drawn growing interests from researchers in computer graphics. Fused deposition modeling (FDM), stereolithographic (SLA), Selective Laser Melting (SLM) and Selective Laser Sintering (SLS) and the four most popular means of 3D printing techniques. Although 3D printing has seen its applications in producing arbitrarily intricate 3D models, the price of the printing materials, especially for those with high quality, are still outrageously high. Therefore, it is desirable to reduce materials used in the fabrication process. Note that this is also a critical operation for reducing production time and therefore the total production cost. For this purpose, an efficient method is to minimize the support structures, which are removed in the post-processing phase of the fabrication task.

As for minimizing support structures, Autodesk R MeshMixerTM provides a semiautomatic orientation optimization tool to minimize support volume, support area, structural strength, or a combination of these three attributes. However, it requires professional experience in setting the geometric parameters manually. A number of literatures have studied various factors that influence the volume of supports, e.g., optimizing the topology of the support structure \cite{DumasHL14,VanekGB14}, determining an optimal fabrication direction \cite{Zhang:2015,HildebrandBA13,padhye2011multi}, partitioning any given model into a set of separate parts that satisfy particular geometric properties such as being pyramidical \cite{Hu_siga14}, has minimal packing volume \cite{VanekGBMCSM14} or inter-lockable \cite{SongFLF15}. However, the partition results of these methods do not simultaneously respect the geometric properties and the support-free printability of the portioned parts. Further, no existing methods ever consider the problem of partitioning a boundary-represented mesh into the least number of parts whose fabrication is free of support structures. The least number of portioned parts corresponds to the least number of seams on the final assembled model, which ensures a nice aesthetics preservation of the model surface; and the support-free fabrication saves material to the most extent, which is particularly helpful in printing objects made of metal powder, resin, or nice plastics, etc.

This paper addresses the problem of decomposing a 3{D} model into the least number of parts, each of which, when printed in a proper direction, is free of support materials. Unfortunately, this problem is identical to the multi-container packing problem, which has been shown to be NP-hard \cite{Fukunaga:2007}. Our solution of the problem draws inspiration from the 1{D} representation of organic models, i.e., skeletons: the topology variations of a natural model can be well-encoded by its skeleton, and a segment of the skeleton corresponding to a chunk of a mesh; further, the mesh chunk is typically a cylinder-like shape which can be printed free of support structures if the printing direction is parallel to the skeletal direction.

We restrict our focus on organic or man-made models since those shapes merit well-defined skeletons. Further, support structures are required for the interior and exterior surface of a mesh model during the 3{D} printing process. Our approach assumes that the interior of a mesh model is hollowed and the mesh model is shelled with a printer-friendly thickness. Therefore, our objective is to partition a model according to the growth of its skeleton into a set of sub-parts, such that each sub-part is represented by a skeletal subgraph which can be fabricated in a good printing direction with no support structure needed. In the meanwhile, we also look for a best set of partitioned subparts which induce the minimal partition length.

Formally, given a printing direction, if the angle formed by the normal of a facet and the printing direction is greater than or equal to $\theta$ which is a printer-dependent value ($60^{\circ}$ for a typical FDM printer), then the facet can be printed without using any support structure. This inspires us to compute a minimum set of subgraph of the skeleton, such that each edge in any subgraph subtends to an axis (the printing direction) by an angle of greater than or equal to $\theta$, the corresponding chunk of the mesh is therefore support-free as printed along this axis. In general, a cone of axes satisfies this angle constraint. However, the volume of the chunk should be within the working space of the printer. Further, if the center of mass of the chunk diverts from the support center too much, e.g., the gravitational torque applied at the mass center is too far away from the center of support that it bends the printing model by a layer of thickness, then the surface quality of the model is poor or the printing task fails since the next printing layer cannot be firmly attached to the previous layer. Our method handles all these issues in a unified Monte Carlo framework which simultaneously looks for the best set of subgraphs which is support-free and with minimal partition length while guaranteeing the extracted subgraphs to have the desired gravitational properties.

In short, our method makes the following contributions:

\begin{itemize}
\item {We partition any given mesh model into the least number of parts that are printable free of support structures; meanwhile, we preserve the aesthetics of the surface with the least number of seams whose length is minimized.}
\item {We propose a Monte Carlo graph partition method based on the guide of 1D Laplacian skeleton of a given mesh model, which simultaneously account for the minimal length and gravitational constraints.}
\end{itemize}
