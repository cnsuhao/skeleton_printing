\section{Overview}

{youyi's comment for the following part: Here an intuitive overview of the problem  and the method is required. e.g., our method focuses on shapes which merit well-defined skeletons, for example, articulated shapes, as in such a setting,}

Observe that the topology of most natural life forms such as trees and animals can be described by their skeletons. Compared to natural skeletons, the medial axis can describe the topology of a mesh model more precisely \cite{ZhangXWYTW15}. However, a medial axis of a 3D mesh model is a 2D surface which cannot be conveniently applied to describe the critical topology changes of the model. Additionally, the medial axis consists of intersecting pieces of planes and conic surfaces, presenting significant complications to algorithms that attempt to construct 3D medial axes.
Reeb graph provides a possible choice for 1D skeleton. During the generation process of any reeb graph, the slicing direction and the position of the representative node on each slide (a connected region) seriously influence the choice of critical points and therefore generation of the Reeb graph. However, the determination of suitable slicing direction and representative nodes is an intractable problem.
Gown by shrinking the mesh model using Laplacian smoothing, 1D Laplacian skeleton provides an excellent choice for reasonably describing the topology of any 3D model \cite{AuTCCL08}. See Figure XXX for an illustration of the Laplacian skeleton.

In nature, the geometric features of most organic models are cylinder-liked, e.g., arms, legs, etc. Therefore, an organ segment can be represented by its corresponding 1D skeleton piece. In the remainder of the paper, by model we mean an organic mesh model of natural life form or cartoon figures preserving nice topology features of real lives. Based on this property, a chunk of a mesh model can be fabricated free of support if and only if its corresponding skeleton piece subtends to the building direction by an angle of no large than ?, where ? is a threshold value determined by experiments.

Problem Statement: Decomposing the 1D Laplacian skeleton of the model into the least support-free subgraphs leads to a partition of the model into the least printable parts free of support structures and cracks on the final assembly model. In addition, since support structures result in bumpy supported areas, support-free fabrication also means a nice preservation of the surface quality of the parts. Further, a minimization of the number of cuts and the total cutting length means a minimum amount of seams and their lengths on the assembled model. Therefore, we shall focus on these two problems in this paper.

The hardness of the problem: Consider an instant of 1D Laplacian skeleton that is a fork with n vectors sharing a common origin; our objective is partitioning the fork into the least number of sub-forks such that each sub-fork can be packed into a cone of angle 2?. This problem is exactly the problem of packing n items with weights w1, ..., wn into bins of capacity c such that all items are packed into the fewest number of bins, which has been shown to be NP-hard \cite{Fukunaga:2007}.

High level of our approach: we shall formulate the problem as an optimization system with both the objectives of the total number of cuts and the cutting length, under the constraint of printing angle of each branch with respect to the build platform, the angle between a cut plane and the printing direction, the dimension of each printed model with respect to the printing space of a given printer, and the area of the base of a printed model. We shall show that the problem is NP-hard; based on a set of training data, we propose a randomized Monte Carlo method to solve the optimization system.
