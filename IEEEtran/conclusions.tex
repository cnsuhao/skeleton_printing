\section{Conclusion}

In this paper, we present a skeleton-based approach for partitioning a shell model into parts which are free of supporting structure when fabricated. We formulate the model partition problem as a constrained graph partition problem which is particularly tailored toward 3{D} fabrication. To tackle the NP-hardness of the problem, we exploit a randomized Monte Carlo method which adaptively searches for better partition possibilities while avoiding local minima. Compared with existing partition-based methods, the advantages of our partition method are as follows:

\begin{itemize}
 \item The models are support-free, especially for the 3D printing techniques including SLA, SLM and SLS. For FDM technique, it requires a bed of support that consumes very little volume of materials.
\item The seams on the assembled model are minimized in terms of cut number and cut length.
\end{itemize}

The support-free feature of our partition approach saves a significant amount of time and printing materials both inside and outside the model of a shell model. Finally, our method is efficient and applicable to a large set of natural and man-made models.
