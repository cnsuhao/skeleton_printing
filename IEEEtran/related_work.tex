\section{Related Work}

\textbf{3D Printing.} In recent, an increasing body of research work has been devoted to 3{D} printing as the emergence of advanced 3{D} printing devices. In computer graphics, a number of literatures have focused on the fabrication of 3{D} models using 3{D} printers. Optimization works have been devoted to structural designs with emphasis on saving printing materials while preserving certain strength \cite{StavaVBCM12,ZhouPZ13,WangWYLTTDCL13,Umetani:2013:CSA,LuSZWFCSTCC14}. The modeling of some particular features have also been studied, for example, deformation behavior \cite{SkourasTCBG13}, animated mechanical characters \cite{CorosTNSFSMB13,CeylanLMAP13}, articulated models with mobile joints \cite{BacherBJP12,CaliCAKSKW12}, models spinnable motions \cite{Bacher14}, and self-balancing \cite{PrevostWLS13}.


\textbf{Model Partition for 3{D} Printing.} A 3{D} printer cannot directly print a model whose size is larger than the printer's working space. To overcome this practical limitation, Luo et al. \cite{LuoBRM12} proposed a solution to partition a given 3D model into parts for 3D printing and then assemble the parts together. This approach has a few advantages: (i) it is cost-effective in the sense that we only need to print a replacement part for a corresponding broken part; (ii) it is convenient for storage and transportation; (iii) changing some parts of a model allows innovative designs. Along this line of research, Hao et al. \cite{hao2011efficient} partitioned a large complex model into simpler 3D printable parts by using curvature-based partitioning. Hildebrand et al. \cite{HildebrandBA13} addressed the directional bias issue in 3D printing by segmenting a 3D model into a few parts each of which is assigned an optimal printing orientation. Vanek et al. \cite{VanekGBMCSM14} reduced the time and material cost of 3D printing by hollowing a 3D model into shells and breaking them into parts, a number of parameters including the total connecting area and volume of each segment are considered during the optimization process. Song et al. \cite{SongFLF15} recently developed a novel voxelization-based approach to construct inter-locking 3D parts from a given 3D model. Without using any glue, Xin et al. \cite{XinLFWHC11} and Song et al. \cite{SongFC12} take a 3D interlocking approach to construct and connect printed 3D parts to form an object assembly. \ca{All these methods do not consider the problem of supporting-free fabrication. Most recently, Hu et al. \cite{Hu_siga14} proposed an nice algorithm for decomposing a solid model into the least number of pyramids, each of which can be fabricated without supporting material. However, their algorithm is only designed for volumetric models while our algorithm aims for both shell and solid models.}


\textbf{Shape Decomposition.} In geometry processing, decomposing a shape into meaningful parts is a fundamental problem. Many research efforts have been devoted to the problem of model decomposition. An excellent survey can be found in \cite{Shamir08}. A large body of research are focusing on partitioning a given 3{D} model into parts which agree with human perception \cite{KatzT03,KatzLT05,JiLCW06,LiuZ07,Golovinskiy:2008,ChenGF09,KaickFKAC14}, among which geometric features captures shape concaveness are mostly exploited in accordance with the minima rules \cite{hoffman1984parts,hoffman1997salience}. Other approaches are more application-oriented. For example, texture mapping techniques often require the input shape being decomposed into flat charts which can be flatten with image textures \cite{zhou2004iso,Garcia:2008:IIG}. Our method of partition are based on shape skeletons. While there have been previous work on skeleton based shape decomposition \cite{lien2006simultaneous,reniers2007skeleton,AuTCCL08}, none of them are tailored towards support-free fabrication.


